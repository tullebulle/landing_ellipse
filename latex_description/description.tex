\documentclass{article}
\usepackage{amsmath}
\usepackage{amssymb}
\usepackage{graphicx}
\usepackage{amsfonts}

\author{Ulrik Unneberg}

\begin{document}
\title{Landing an ellipse}
\maketitle

\section*{Problem Setup}

Consider an ellipse with semi-major axis \( a = 1.0 \) and semi-minor axis \( b = 0.5 \), density \( \rho_s = 1.2 \), and moment of inertia \( I = 1.0 \). The surrounding fluid has density \( \rho_f = 1.0 \) and gravity \( g = 9.81, \ \text{m/s}^2 \). Derived non-dimensional parameters are as follows:

\[
\beta = \frac{b}{a}, \quad \rho_* = \frac{\rho_s}{\rho_f}
\]

Forces and torque parameters are given by
\begin{itemize}
    \item Constants: \( A = 1.4 \), \( B = 1.0 \)
    \item Viscous terms: \( \mu = 0.2 \), \( \nu = 0.2 \)
    \item Torque constants: \( C_T = 1.2 \), \( C_R = \pi \)
\end{itemize}

The system is simulated with a time step \( \Delta t = 0.01 \) and total time \( T = 5 \).

\section*{Dynamics Equations}

Define the translational velocities \( u \) and \( v \), the angular velocity \( w \), and orientation \( \theta \). The forces \( F_u \), \( F_v \) and moment \( M \) are computed as:

\[
F = \frac{1}{\pi} \left( A - B \frac{u^2 - v^2}{u^2 + v^2 + \epsilon} \right) \sqrt{u^2 + v^2 + \epsilon}
\]

\[
M = 0.2 (\mu + \nu |w|) w
\]
\[
\Gamma = \frac{2}{\pi} \left( C_R w - C_T \frac{u v}{\sqrt{u^2 + v^2 + \epsilon}} \right)
\]

where \( \epsilon \) is a small constant to avoid division by zero.

\section*{Equations of Motion}

The equations of motion governing \( u \), \( v \), and \( w \) are given by:

\[
\dot{u} = \frac{(I + \beta^2) v w - \Gamma v - \sin \theta - Fu}{I + \beta^2}
\]
\[
\dot{v} = \frac{-(I + 1) u w + \Gamma u - \cos \theta - Fv}{I + 1}
\]
\[
\dot{w} = \frac{-(1 - \beta^2) u v - M}{0.25 \left( I (1 + \beta^2) + 0.5 (1 - \beta^2)^2 \right)}
\]

The position \( (x, y) \) and orientation \( \theta \) are updated as follows:

\[
x \leftarrow x + (u \cos \theta - v \sin \theta) \Delta t
\]
\[
y \leftarrow y + (u \sin \theta + v \cos \theta) \Delta t
\]
\[
\theta \leftarrow \theta + w \Delta t
\]

\section{Reinforcement Learning}
Our objective is the following 
\begin{itemize}
    \item $theta_G = \pi/4$
    \item $x_G = 100$
    \item $y_G = -50$
\end{itemize}
We want to model the torque action is constant control torque
$$
\tau_t = \tanh(a_t) \in \{-1, 1\}.
$$



\section*{Numerical Simulation}

The system is numerically simulated over a fixed time period with a time step of \( \Delta t = 0.01 \). Initial conditions are:
\[
x = 0, \quad y = 0, \quad \theta = \frac{\pi}{4}, \quad u = 0, \quad v = 0, \quad w = 0
\]

\section*{Results}

The trajectory of the ellipse can be visualized by plotting \( x \) vs \( y \), showing the dynamics influenced by the forces and torques as described.




\end{document}